\documentclass[10pt,a4paper]{extarticle}
\pagenumbering{gobble}
\usepackage[left=0.6in,top=0.6in,right=0.6in,bottom=0.6in]{geometry}
\usepackage{hyperref}
\usepackage{titlesec}
\usepackage{enumitem}
\usepackage{graphicx}

\titleformat{\section}{\large\bfseries\scshape\raggedright}{}{0em}{}[\titlerule]
\titlespacing*{\section}{0pt}{12pt}{8pt}

\begin{document}

\begin{center}
  \begin{minipage}{\textwidth}
    \centering
    {\LARGE\textbf{Eyup Bayram}} \hspace{2pt} {\LARGE Backend Engineer}\\[8pt]
    Umraniye, Istanbul \textbullet\ 
    \href{mailto:egorkemb@gmail.com}{egorkemb@gmail.com} \textbullet\
    +90 505 554 84 58 \textbullet\
    \href{https://linkedin.com/in/egbayram}{linkedin.com/in/egbayram} \textbullet\
    \href{https://github.com/egbay}{github.com/egbay}
  \end{minipage}
\end{center}

\section{Summary}
Backend-heavy software engineer (7+ yrs) focused on AI products, and cloud-native architectures with hands-on backend experience (Python/FastAPI, Node/NestJS, PostgreSQL). Designed reliable billing pipelines and low-latency APIs on AWS Lambda. Looking to drive platform reliability and developer velocity in a backend role. Passionate about leveraging technology to solve real-world problems and contribute to impactful projects. 

\section{Experience}
\textbf{\href{https://linkedin.com/in/egbayram}{Freelancer} \textit{(Istanbul, Türkiye)}} \hfill Remote\\
\textit{Software Engineer} \hfill July 2024 -- Present
\begin{itemize}[leftmargin=*,noitemsep,topsep=0pt]
    \item \textbf{Full-Stack Development for Multiple AI Products}: Defined service boundaries, API contracts, and inter-service communication (HTTP/queues) in a microservice architecture. Used FastAPI, Django, Node.js, TypeScript, Prisma across products. Built event-driven workflows and serverless functions on AWS Lambda. Rapidly shipped UI with Vercel V0 and hardened React components for production. Used Cloudflare workers, AWS Step functions, AWS Lambda, AWS API Gateway, Vercel. 
    \item \textbf{Payment \& Billing + Online Learning Platform}: Modeled DB with Prisma; managed schema migrations and transactions; implemented repository/service layers to ensure data integrity. Stack: NestJS, TypeScript, Prisma, PostgreSQL. Deployed on AWS, using EC2, RDS. 
    \item \textbf{AI Fashion Project}: Developed REST APIs for model invocations, job management, and status tracking using FastAPI. Integrated Supabase authentication \& storage; enforced signed URLs and access policies. Built async workflows and queuing for long-running jobs. Hands-on backend work with FastAPI, PostgreSQL, several GCP services, GitHub Actions, Celery, RabbitMQ.
\end{itemize}


\noindent\textbf{\href{https://smartex.ai}{Smartex AI} \textit{(Porto/Istanbul, Portugal/Türkiye)}} \hfill Remote\\
\textit{Backend Engineer \& Managing Director} \hfill September 2022 --  July 2024
\begin{itemize}[leftmargin=*,noitemsep,topsep=0pt]
    \item Led Smartex’s expansion in Türkiye post-acquisition of Tuvis AI, managing budgets, compliance, and resourcing.
    \item Backend development with Node.js, TypeScript, PostgreSQL, MongoDB, TypeORM, Prisma; key role in migration from TypeORM to Prisma.
    \item Owned production support: created API endpoints, resolved live bugs, and improved operational stability.
\end{itemize}


\noindent\textbf{\href{https://www.tuvisai.com}{Tuvis AI} \textit{(Istanbul, Türkiye)}} \hfill Istanbul, Türkiye\\
\textit{Co-Founder \& Software Engineer} \hfill July 2018 -- September 2022
\begin{itemize}[leftmargin=*,noitemsep,topsep=0pt]
    \item Co-founded and scaled AI-driven anomaly detection solutions for manufacturing process optimization.
    \item Led a team across project planning, technical implementation, and server management.
    \item Hands-on backend work with Flask, Django, PostgreSQL, Redis, AWS (S3, EC2, RDS, Lambda, CloudWatch, ECR), GitHub Actions, Celery, RabbitMQ.
    \item Engaged clients with technical insights; secured multiple R\&D projects funded by TUBITAK, KOSGEB, and Growth Circuit VC.
\end{itemize}

\noindent\textbf{\href{https://www.peakode.com}{Peakode} \textit{(Bursa, Türkiye)}} \hfill Bursa, Türkiye\\
\textit{Research \& Development Engineer} \hfill September 2016 -- June 2017
\begin{itemize}[leftmargin=*,noitemsep,topsep=0pt]
    \item Built IoT solutions with XBee sensor networks, xCTU, Node.js, and Raspberry Pi.
\end{itemize}

\noindent\textbf{\href{https://www.tofas.com.tr}{TOFAS} \textit{(Bursa, Türkiye)}} \hfill Bursa, Türkiye\\
\textit{Research \& Development Engineer} \hfill September 2016 -- June 2017
\begin{itemize}[leftmargin=*,noitemsep,topsep=0pt]
    \item Conducted CAN Bus analysis and contributed to Instrument Cluster Panel development for an EV prototype using C++ and GLStudio.
    \item Awarded ``Best R\&D Project'' within TOFAS.
\end{itemize}


\section{Education \& Certifications}
\textbf{Kocaeli University} \hfill Kocaeli, Türkiye\\
\textit{M.Sc. in Computer Engineering} \hfill Jan 2019 -- Thesis Stage\\[4pt]
\textbf{Emerald Cultural Institute} \hfill Dublin, Ireland\\
\textit{Academic English Program (IELTS 6.5)} \hfill Oct 2017 -- Jun 2018\\[4pt]
\textbf{Uludag University} \hfill Bursa, Türkiye\\
\textit{B.Sc. in Electrical \& Electronics Engineering} \hfill Sep 2013 -- Jun 2017\\[6pt]


\section{Skills}
\textbf{Languages:} Python, JavaScript/TypeScript\\
\textbf{Databases:} MySQL, PostgreSQL, MongoDB\\
\textbf{Frameworks:} Flask, FastAPI, Django, Node.js, NestJS, Prisma, TypeORM\\
\textbf{Cloud/Tech:} Serverless Framework, AWS (S3, SNS, EC2, API Gateway, Lambda, RDS, CodeCommit, CloudWatch)\\
\textbf{Other:} Celery, Redis, RabbitMQ, GitHub Actions, Sentry, CAN Bus, MQTT\\
\textbf{Language Proficiency:} English (IELTS 6.5)

\section{Volunteer}
Co-founded the IEEE Bursa Uludag University Student Branch; organized events, workshops, and technical trainings to help students build leadership and engineering skills.

\end{document}
